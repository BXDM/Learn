\documentclass[UTF8]{ctexart}
\usepackage{amsmath}
\usepackage[T1]{fontenc}
\usepackage{mathtools}    % loads »amsmath«

\title{位场理论}
\author{北巷的猫}
\begin{document}
\maketitle           %生成title
北巷的猫

   Maxwell's equations (Differential Form):
   \begin{subequations}
     \begin{alignat}{2}
       \mathllap{\text{高斯定理}\qquad} && \nabla \times \emph{\textbf{H}} &=\sigma \emph{\textbf{E}} + \varepsilon   \frac{\partial\emph{\textbf{E}}}{\partial t} \\
       \mathllap{\text{高斯定理}\qquad} && \nabla \times \emph{\textbf{E}} &=-\mu  \frac{\partial\emph{\textbf{H}}}{\partial t} \\
       \mathllap{\text{法拉第定律}\qquad} && \nabla \cdot \emph{\textbf{H}} &=0 \\
       \mathllap{\text{安培定理}\qquad} && \nabla \cdot \emph{\textbf{E}} &=0
     \end{alignat}
   \end{subequations}

北巷的猫$\vec{n}$,$h_A$
 Maxwell's equations (Differential Form):
   \begin{subequations}
     \begin{alignat}{2}
       \nabla \times \emph{\textbf{H}}=\sigma \emph{\textbf{E}} + \varepsilon   \frac{\partial\emph{\textbf{E}}}{\partial t} \\
        \nabla \times \emph{\textbf{E}}=-\mu  \frac{\partial\emph{\textbf{H}}}{\partial t} \\
        \nabla \cdot \emph{\textbf{H}}=0 \\
        \nabla \cdot \emph{\textbf{E}}=0
     \end{alignat}
   \end{subequations}

\end{document}
