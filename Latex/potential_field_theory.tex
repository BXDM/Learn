\documentclass[UTF8]{ctexart}       %引入向入中文库
\bibliographystyle{plain}           %设置参考文献引用格式
\usepackage{amsmath,esint}    %bigtriangledown所需要使用到的数学公式宏
\usepackage{mathtools}        %定界符需要的宏包支持
\usepackage{color}            %调用颜色包
\author{北巷的猫}
\title{位场理论}
\begin{document}
\maketitle              %生成标题
\pagebreak              %在此处分页
\tableofcontents        %生成目录
\pagebreak              %在此处分页

\section{场的定义}
    \subsection{场的定义}
    {\heiti 数学上的定义:}

    A field may be defined mathematically as the function of a set of variables in a given space。

    给定空间内一系列变量的函数

    a field is a distribution in space of any quantity: scalar, vector, time  dependent, or independent of time \\

    {\heiti 物理上的定义}

    A field is a physical quantity, represented by a number or tensor, that has a value for each point in space and time!

    场是一个用数字或张量表示的,在时间和空间上任意一点都有值的物理量。



    \subsection{位场的定义}

   设矢量场在区域D内的场量函数为:$\vec{A}(x,y,z)$,对于区域D内的任一闭合曲线L,都有$\oint_L A \cdot dl = 0$,则称矢量场$\vec{A}(x,y,z)$为位场。

   根据stokes定理:
   $$\oint A \cdot dl = \iint_s \nabla \times A \cdot dS $$
   得到:$\oint A \cdot dl = 0$,即矢量场A的积分与路径无关,即存在未函数,使得:
   $$ A = -\nabla\dot U $$
   {\color{cyan}即无旋为位场。}     %使用蓝色字体显示

    \subsection{场的刻画}

        \subsubsection{梯度}
        
        $$\nabla\varphi = grad\varphi = \frac{\partial\varphi}{\partial x}\vec {i}+ \frac{\partial\varphi}{\partial y}\vec {j} + \frac{\partial\varphi}{\partial z}\vec {k}$$
        
        \subsubsection{散度}
        
        $$\nabla \cdot \vec{A} = div\vec{\varphi}=
\frac{\partial\varphi}{\partial A_x} + \frac{\partial\varphi}{\partial A_ y} + \frac{\partial\varphi}{\partial A_z}$$
        
        \subsubsection{旋度}
        
        \[\nabla \times \vec{A} = rat \vec{A} =\begin{bmatrix}
                                           \vec{i} & \vec{j} & \vec{k} \\
                                           \frac{\partial}{\partial x} & \frac{\partial}{\partial x} & \frac{\partial}{\partial x} \\
                                           A_ x & A_ y & A_ z
                                         \end{bmatrix} \]
        
\section{场的基本特征}
    \subsection{源}
    \subsection{位满足的方程}
    \subsection{位的分类}
    \subsection{磁偶极子产生的位}
\section{几个重要的场}
    \subsection{引力场}

1.{\heiti万有引力定律} \\
$$ \overrightarrow{\mathbf{F_{12}}} = \mathbf{-f\frac{m_1m_2}{r_{12} ^3}}\overrightarrow{\mathbf{r_{12}}}$$

2.{\heiti引力场的涡旋特征}
$$ \oint\limits_L\overrightarrow{\mathbf{G}}\mathbf{\cdot d\overrightarrow {l}} = 0$$
根据斯托克斯公式
$$\oint\limits_L \overrightarrow{G}\cdot d\overrightarrow{l} = \iint\limits_S rot\overrightarrow{G}\cdot d\overrightarrow{s}$$
$$\nabla \times \overrightarrow{G}\cdot d\overrightarrow{s}$$
它说明引力场$\overrightarrow{G}$是处处无旋的。\\
因此,引入标位
$$ \mathbf{U=(x,y,z)=\int _{(x_0,y_0,z_0)}^{(x,y,z)}\overrightarrow{G}\cdot d\overrightarrow{l}} $$


    \subsection{恒定电流场}
\section{位场变换}
    \subsection{δ函数的定义}

狄拉克函数(Dirac Delta Function) \\

{\heiti 定义}     %使用黑体
$$\delta(x) =
\begin{dcases}
+\infty,&x = 0 \\
 0,&x \neq 0
\end{dcases}$$

{\heiti 性质}  \\
$$\int_{+\infty}^{-\infty} \delta(x)\mathrm{d}x = 1$$


    \subsection{格林公式}

1.格林第一公式\cite{GreenEquation}    % \cite表示在这里引用参考文献

$$\oiint\limits_S V\frac{\partial U}{\partial n}\mathrm{d}s     %limits使下标显示在正下方
= \iiint\limits_V \left(V\nabla^ 2 U+\nabla U \nabla U)
\right)\mathrm{d}\upsilon$$

2.格林第二公式
$$\oiint\limits_S \left(V \frac{\partial U}{\partial n} - U\frac{\partial V}{\partial n}\right)\mathrm{d}s
=\iiint\limits_V \left(V \nabla^2 U - U \nabla^2 V \right)\mathrm{d}\upsilon$$

    \subsection{调和函数}
    \subsection{泊松公式}
    \subsection{延拓}
    \subsection{基本位场变换公式}
    \subsection{傅里叶变换}
    \subsection{磁化极}


\bibliography{potential_field_theory}      %在此处打印参考文献,参考文献库为 potential_field_theory.bib





\end{document}
