\documentclass{article}
\usepackage{CJK}
%美国数学学会的三个数学包
\usepackage{amssymb,amsfonts,amsmath,amsthm}
\usepackage[dvips]{graphicx}
\usepackage{overpic}
\usepackage{subfigure}
%加入代码的灰色底纹
\usepackage{color}
\usepackage{framed}
\definecolor {shadecolor}{rgb}{0.92,0.92,0.92}
%一级子标题居中效果
%\usepackage[center]{titlesec}
%\titleformat*{\section}{\centering}
\usepackage{listings}%插入代码的图
% 注意,代码中不能含有中文,否则为无法编译。
\lstset{
   %行号
  numbers=left,
   %背景框
   framexleftmargin=10mm,
    frame=none,
   %背景色
   %backgroundcolor=\color[rgb]{1,1,0.76},
   backgroundcolor=\color[RGB]{245,245,244},
   %样式
   keywordstyle=\bf\color{blue},
   identifierstyle=\bf,
   numberstyle=\color[RGB]{0,192,192},
   commentstyle=\it\color[RGB]{0,96,96},
   stringstyle=\rmfamily\slshape\color[RGB]{128,0,0},
   %显示空格
   showstringspaces=false
   escapeinside=``}


\usepackage[colorlinks,citecolor=red,CJKbookmarks=true]{hyperref}%红色网址引用
\usepackage{mathrsfs}%对于花体的支持
\usepackage[top=1in, bottom=1in, left=1.25in, right=1.25in]{geometry}


\newtheorem{thm}{定理}
\newtheorem{fthm}[thm]{Fundamental Theorem}
\newtheorem{dfn}[thm]{定义}
\newtheorem{lem}[thm]{引理}
\newtheorem{cor}[thm]{推论}
\newtheorem{exa}[thm]{例}


\begin{document}
\begin{CJK}{GBK}{song}
\newtheorem{theorem}{定理}[section]
\newtheorem{assumption}{假设}[section]
\newtheorem{corollary}{推论}[section]
\newtheorem{proposition}{命题}[section]
\newtheorem{lemma}{引理}[section]
\newtheorem{definition}{定义}[section]
\newtheorem{assum}{假设}[section]
\newtheorem{algo}{算法}[section]
\newtheorem{remark}{注}[section]






\title{\centering\textbf{论文题目}}
\author{作者1, 作者2\thanks{通讯作者\&邮箱:****@mail.***.edu.cn}, 作者3}
\date{}
\maketitle
{\bf 摘要:} 摘要内容
\vspace{0.5cm}


\begin{flushleft}
{\bf 关键词:} 关键词;关键词;关键词;关键词;关键词;关键词.
\par
\end{flushleft}
\vspace*{12mm}%新一页的空行%




\section{引言}


引言……






\vspace*{12mm}%新一页的空行%


\section{第二小节}


第二小节内容……




%%%%%%%%%%%%%%%%%%%%%%%%定理(其他引理、注释同理)%%%%%%%%%%%%%%%%%
\begin{theorem}
定理内容……
\end{theorem}


\begin{proof}
证明过程
\end{proof}



\vspace*{12mm}%新一页的空行%

%%%%%%%%%参考文献%%%%%%%%%%%%%%%
\begin{thebibliography}{1} 
\bibitem{1}孙志忠,袁慰平,闻震初:数值分析,\emph{东南大学出版社}(2011). 
\end{thebibliography}
\begin{thebibliography}{1} \bibitem{1}孙志忠,袁慰平,闻震初:数值分析,\emph{东南大学出版社}(2011). \end{thebibliography}



\end{CJK}

\end{document}


